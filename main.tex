\documentclass{article}
\usepackage[utf8]{inputenc}
\usepackage{url}
\usepackage{listings}
\newcommand{\pyspeckit}{\texttt{pyspeckit}}
\begin{document}


Pyspeckit is a tool and library for spectroscopic analysis. 


\section{Introduction}
Spectroscopy is an important tool for astronomy.

\section{Graphical Design Choices}
\subsection{GUI development}
Many astronomers are familiar with IRAF's \texttt{splot} tool, which is useful
for fitting Gaussian profiles to spectral lines.  It used keyboard interaction
to specify the fitting region and guesses for fitting the line profile, but for
most users, gave them access to those results \emph{only} through the GUI.

\texttt{pyspeckit}'s fitting GUI was built to match \texttt{splot}'s
functionality, but with addition means of interacting with the fitter.  In
\texttt{splot}, reproducing any given fit is challenging, since subtle changes
in the cursor position can significantly change the fit result.  In \pyspeckit,
it is possible to record the results of fits programatically and re-fit using
those same results.

The GUI was built using \texttt{matplotlib}'s canvas interaction tools.  These
tools are limited to GUI capabilities that are compatible with all platforms
(e.g., Qt, Tk, Gtk) and therefore exclude some of the more sophisticated fitting
tools found in other software (e.g., \texttt{glue}).

\subsection{Plotting}
Plotting in \pyspeckit is meant to provide the shortest path to
publication-quality figures.  The default plotting mode uses histogram-style
line plots and labels axes with \latex-formatted versions of units.


\end{document}
